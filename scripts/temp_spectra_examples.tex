\documentclass[]{article}
\usepackage{lmodern}
\usepackage{amssymb,amsmath}
\usepackage{ifxetex,ifluatex}
\usepackage{fixltx2e} % provides \textsubscript
\ifnum 0\ifxetex 1\fi\ifluatex 1\fi=0 % if pdftex
  \usepackage[T1]{fontenc}
  \usepackage[utf8]{inputenc}
\else % if luatex or xelatex
  \ifxetex
    \usepackage{mathspec}
  \else
    \usepackage{fontspec}
  \fi
  \defaultfontfeatures{Ligatures=TeX,Scale=MatchLowercase}
\fi
% use upquote if available, for straight quotes in verbatim environments
\IfFileExists{upquote.sty}{\usepackage{upquote}}{}
% use microtype if available
\IfFileExists{microtype.sty}{%
\usepackage{microtype}
\UseMicrotypeSet[protrusion]{basicmath} % disable protrusion for tt fonts
}{}
\usepackage[margin=1in]{geometry}
\usepackage{hyperref}
\hypersetup{unicode=true,
            pdftitle={Estimating Chi - examples},
            pdfborder={0 0 0},
            breaklinks=true}
\urlstyle{same}  % don't use monospace font for urls
\usepackage{color}
\usepackage{fancyvrb}
\newcommand{\VerbBar}{|}
\newcommand{\VERB}{\Verb[commandchars=\\\{\}]}
\DefineVerbatimEnvironment{Highlighting}{Verbatim}{commandchars=\\\{\}}
% Add ',fontsize=\small' for more characters per line
\usepackage{framed}
\definecolor{shadecolor}{RGB}{248,248,248}
\newenvironment{Shaded}{\begin{snugshade}}{\end{snugshade}}
\newcommand{\KeywordTok}[1]{\textcolor[rgb]{0.13,0.29,0.53}{\textbf{#1}}}
\newcommand{\DataTypeTok}[1]{\textcolor[rgb]{0.13,0.29,0.53}{#1}}
\newcommand{\DecValTok}[1]{\textcolor[rgb]{0.00,0.00,0.81}{#1}}
\newcommand{\BaseNTok}[1]{\textcolor[rgb]{0.00,0.00,0.81}{#1}}
\newcommand{\FloatTok}[1]{\textcolor[rgb]{0.00,0.00,0.81}{#1}}
\newcommand{\ConstantTok}[1]{\textcolor[rgb]{0.00,0.00,0.00}{#1}}
\newcommand{\CharTok}[1]{\textcolor[rgb]{0.31,0.60,0.02}{#1}}
\newcommand{\SpecialCharTok}[1]{\textcolor[rgb]{0.00,0.00,0.00}{#1}}
\newcommand{\StringTok}[1]{\textcolor[rgb]{0.31,0.60,0.02}{#1}}
\newcommand{\VerbatimStringTok}[1]{\textcolor[rgb]{0.31,0.60,0.02}{#1}}
\newcommand{\SpecialStringTok}[1]{\textcolor[rgb]{0.31,0.60,0.02}{#1}}
\newcommand{\ImportTok}[1]{#1}
\newcommand{\CommentTok}[1]{\textcolor[rgb]{0.56,0.35,0.01}{\textit{#1}}}
\newcommand{\DocumentationTok}[1]{\textcolor[rgb]{0.56,0.35,0.01}{\textbf{\textit{#1}}}}
\newcommand{\AnnotationTok}[1]{\textcolor[rgb]{0.56,0.35,0.01}{\textbf{\textit{#1}}}}
\newcommand{\CommentVarTok}[1]{\textcolor[rgb]{0.56,0.35,0.01}{\textbf{\textit{#1}}}}
\newcommand{\OtherTok}[1]{\textcolor[rgb]{0.56,0.35,0.01}{#1}}
\newcommand{\FunctionTok}[1]{\textcolor[rgb]{0.00,0.00,0.00}{#1}}
\newcommand{\VariableTok}[1]{\textcolor[rgb]{0.00,0.00,0.00}{#1}}
\newcommand{\ControlFlowTok}[1]{\textcolor[rgb]{0.13,0.29,0.53}{\textbf{#1}}}
\newcommand{\OperatorTok}[1]{\textcolor[rgb]{0.81,0.36,0.00}{\textbf{#1}}}
\newcommand{\BuiltInTok}[1]{#1}
\newcommand{\ExtensionTok}[1]{#1}
\newcommand{\PreprocessorTok}[1]{\textcolor[rgb]{0.56,0.35,0.01}{\textit{#1}}}
\newcommand{\AttributeTok}[1]{\textcolor[rgb]{0.77,0.63,0.00}{#1}}
\newcommand{\RegionMarkerTok}[1]{#1}
\newcommand{\InformationTok}[1]{\textcolor[rgb]{0.56,0.35,0.01}{\textbf{\textit{#1}}}}
\newcommand{\WarningTok}[1]{\textcolor[rgb]{0.56,0.35,0.01}{\textbf{\textit{#1}}}}
\newcommand{\AlertTok}[1]{\textcolor[rgb]{0.94,0.16,0.16}{#1}}
\newcommand{\ErrorTok}[1]{\textcolor[rgb]{0.64,0.00,0.00}{\textbf{#1}}}
\newcommand{\NormalTok}[1]{#1}
\usepackage{graphicx,grffile}
\makeatletter
\def\maxwidth{\ifdim\Gin@nat@width>\linewidth\linewidth\else\Gin@nat@width\fi}
\def\maxheight{\ifdim\Gin@nat@height>\textheight\textheight\else\Gin@nat@height\fi}
\makeatother
% Scale images if necessary, so that they will not overflow the page
% margins by default, and it is still possible to overwrite the defaults
% using explicit options in \includegraphics[width, height, ...]{}
\setkeys{Gin}{width=\maxwidth,height=\maxheight,keepaspectratio}
\IfFileExists{parskip.sty}{%
\usepackage{parskip}
}{% else
\setlength{\parindent}{0pt}
\setlength{\parskip}{6pt plus 2pt minus 1pt}
}
\setlength{\emergencystretch}{3em}  % prevent overfull lines
\providecommand{\tightlist}{%
  \setlength{\itemsep}{0pt}\setlength{\parskip}{0pt}}
\setcounter{secnumdepth}{0}
% Redefines (sub)paragraphs to behave more like sections
\ifx\paragraph\undefined\else
\let\oldparagraph\paragraph
\renewcommand{\paragraph}[1]{\oldparagraph{#1}\mbox{}}
\fi
\ifx\subparagraph\undefined\else
\let\oldsubparagraph\subparagraph
\renewcommand{\subparagraph}[1]{\oldsubparagraph{#1}\mbox{}}
\fi

%%% Use protect on footnotes to avoid problems with footnotes in titles
\let\rmarkdownfootnote\footnote%
\def\footnote{\protect\rmarkdownfootnote}

%%% Change title format to be more compact
\usepackage{titling}

% Create subtitle command for use in maketitle
\newcommand{\subtitle}[1]{
  \posttitle{
    \begin{center}\large#1\end{center}
    }
}

\setlength{\droptitle}{-2em}
  \title{Estimating Chi - examples}
  \pretitle{\vspace{\droptitle}\centering\huge}
  \posttitle{\par}
  \author{}
  \preauthor{}\postauthor{}
  \date{}
  \predate{}\postdate{}


\begin{document}
\maketitle

This notebook imports the temperature spectral data saved from matlab
(from structure \texttt{diss.sclar\_spectra.scalar\_spec}), corrects for
the frequency response of the sensor using Vachon and Lueck correction
(used by Rockland in odas) and fits the model stated in Bluteau to the
temperature spectrum to estimate \(\chi\).

Data from profile\_nr 32 was used for testing, dive depth was 414 m. 146
spectra were calculated.

\subsubsection{Correcting for frequency
response}\label{correcting-for-frequency-response}

\begin{Shaded}
\begin{Highlighting}[]
\CommentTok{# correction as in odas library}
\CommentTok{# time constant for double-pole response:}
\NormalTok{F_}\DecValTok{0}\NormalTok{ <-}\StringTok{ }\DecValTok{25}\OperatorTok{*}\KeywordTok{sqrt}\NormalTok{(mean_speed)}
\NormalTok{tau_therm <-}\StringTok{ }\DecValTok{2}\OperatorTok{*}\NormalTok{pi}\OperatorTok{*}\NormalTok{F_}\DecValTok{0} \OperatorTok{/}\StringTok{ }\KeywordTok{sqrt}\NormalTok{(}\KeywordTok{sqrt}\NormalTok{(}\DecValTok{2}\NormalTok{) }\OperatorTok{-}\StringTok{ }\DecValTok{1}\NormalTok{)   }
\NormalTok{tau_therm <-}\StringTok{ }\DecValTok{1} \OperatorTok{/}\StringTok{ }\NormalTok{tau_therm}
\CommentTok{# correction:}
\CommentTok{# create a NA matrix for gradT1 and T2 of correct size and then fill with corrected values}
\NormalTok{gradT1_c <-}\StringTok{ }\KeywordTok{matrix}\NormalTok{(}\DataTypeTok{data=}\OtherTok{NA}\NormalTok{, }\DataTypeTok{nrow=}\KeywordTok{length}\NormalTok{(freq[,}\DecValTok{1}\NormalTok{]), }\DataTypeTok{ncol=}\KeywordTok{length}\NormalTok{(freq[}\DecValTok{1}\NormalTok{,]))}
\NormalTok{gradT2_c <-}\StringTok{ }\KeywordTok{matrix}\NormalTok{(}\DataTypeTok{data=}\OtherTok{NA}\NormalTok{, }\DataTypeTok{nrow=}\KeywordTok{length}\NormalTok{(freq[,}\DecValTok{1}\NormalTok{]), }\DataTypeTok{ncol=}\KeywordTok{length}\NormalTok{(freq[}\DecValTok{1}\NormalTok{,]))}
\ControlFlowTok{for}\NormalTok{(index }\ControlFlowTok{in} \DecValTok{1}\OperatorTok{:}\KeywordTok{length}\NormalTok{(freq[}\DecValTok{1}\NormalTok{,]))\{}
\NormalTok{  gradT1_c[,index] <-}\StringTok{ }\NormalTok{gradT1[,index] }\OperatorTok{*}\StringTok{ }\NormalTok{(}\DecValTok{1} \OperatorTok{+}\StringTok{ }\NormalTok{(}\DecValTok{2}\OperatorTok{*}\NormalTok{pi}\OperatorTok{*}\NormalTok{tau_therm[index]}\OperatorTok{*}\NormalTok{freq[,index])}\OperatorTok{^}\DecValTok{2}\NormalTok{)}\OperatorTok{^}\DecValTok{2}
\NormalTok{  gradT2_c[,index] <-}\StringTok{ }\NormalTok{gradT2[,index] }\OperatorTok{*}\StringTok{ }\NormalTok{(}\DecValTok{1} \OperatorTok{+}\StringTok{ }\NormalTok{(}\DecValTok{2}\OperatorTok{*}\NormalTok{pi}\OperatorTok{*}\NormalTok{tau_therm[index]}\OperatorTok{*}\NormalTok{freq[,index])}\OperatorTok{^}\DecValTok{2}\NormalTok{)}\OperatorTok{^}\DecValTok{2}
\NormalTok{\} }\CommentTok{# end for loop}
\end{Highlighting}
\end{Shaded}

\paragraph{Plot corrected sample
spectra}\label{plot-corrected-sample-spectra}

\includegraphics{temp_spectra_examples_files/figure-latex/unnamed-chunk-4-1.pdf}

\paragraph{\texorpdfstring{Fit model to spectra with \(\chi\) as
parameter estimated with MLE (maximum likelihood
estimation)}{Fit model to spectra with \textbackslash{}chi as parameter estimated with MLE (maximum likelihood estimation)}}\label{fit-model-to-spectra-with-chi-as-parameter-estimated-with-mle-maximum-likelihood-estimation}

model to fit:
\[\Psi_{\delta T/\delta x_i}(k) = C_T \chi \epsilon^{-1/3} k^{1/3}\]

with:

\begin{itemize}
\tightlist
\item
  \(\Psi_{\delta T/\delta x_i}\) = corrected energy spectrum
  (\texttt{gradT\_c}),
\item
  \(C_T\) = Obukhov-Corrsin universal constant with values between 0.3
  and 0.5, 0.4 recommended,
\item
  \(\epsilon\) = dissipation energy estimated from shear probe signal
  (\texttt{eps1},\texttt{eps2}),
\item
  \(k\) = wavenumber (\(k\) is here in rad/m, a conversion coefficient
  to convert to cpm should be added to the model?) (\texttt{waven})
\end{itemize}

\subparagraph{Convert measured spectra to wavenumber spectra by
multiplying with
mean\_speed}\label{convert-measured-spectra-to-wavenumber-spectra-by-multiplying-with-mean_speed}

\begin{Shaded}
\begin{Highlighting}[]
\CommentTok{# make empty (NA) matrix of correct size}
\NormalTok{P_gradT1_c <-}\StringTok{ }\KeywordTok{matrix}\NormalTok{(}\OtherTok{NA}\NormalTok{, }\DataTypeTok{nrow =} \KeywordTok{length}\NormalTok{(gradT1_c[,}\DecValTok{1}\NormalTok{]), }\DataTypeTok{ncol=}\KeywordTok{length}\NormalTok{(gradT1_c[}\DecValTok{1}\NormalTok{,]))}
\NormalTok{P_gradT2_c <-}\StringTok{ }\KeywordTok{matrix}\NormalTok{(}\OtherTok{NA}\NormalTok{, }\DataTypeTok{nrow =} \KeywordTok{length}\NormalTok{(gradT2_c[,}\DecValTok{1}\NormalTok{]), }\DataTypeTok{ncol=}\KeywordTok{length}\NormalTok{(gradT2_c[}\DecValTok{1}\NormalTok{,]))}
\CommentTok{# fill matrix with converted values}
\ControlFlowTok{for}\NormalTok{(index }\ControlFlowTok{in} \DecValTok{1}\OperatorTok{:}\KeywordTok{length}\NormalTok{(gradT1_c[,}\DecValTok{1}\NormalTok{]))\{}
\NormalTok{  P_gradT1_c[index,] <-}\StringTok{ }\NormalTok{gradT1_c[index,] }\OperatorTok{*}\StringTok{ }\NormalTok{mean_speed}
\NormalTok{  P_gradT2_c[index,] <-}\StringTok{ }\NormalTok{gradT2_c[index,] }\OperatorTok{*}\StringTok{ }\NormalTok{mean_speed}
\NormalTok{\}}
\end{Highlighting}
\end{Shaded}

\subparagraph{\texorpdfstring{Find upper limit for \(k\) for model fit
using the criterium correction \(H(k) \leq 3\) and
\(k \leq 0.1 \eta ^{-1}\)}{Find upper limit for k for model fit using the criterium correction H(k) \textbackslash{}leq 3 and k \textbackslash{}leq 0.1 \textbackslash{}eta \^{}\{-1\}}}\label{find-upper-limit-for-k-for-model-fit-using-the-criterium-correction-hk-leq-3-and-k-leq-0.1-eta--1}

\begin{Shaded}
\begin{Highlighting}[]
\CommentTok{# initialise K_max}
\NormalTok{K_max <-}\StringTok{ }\KeywordTok{c}\NormalTok{()}
\CommentTok{# loop through all spectra in profile 32}
\ControlFlowTok{for}\NormalTok{(segment }\ControlFlowTok{in} \DecValTok{1}\OperatorTok{:}\DecValTok{146}\NormalTok{)\{}
  \CommentTok{# calculate correction factor Hf}
\NormalTok{  f <-}\StringTok{ }\NormalTok{freq[,segment]}
\NormalTok{  tau0 <-}\StringTok{ }\FloatTok{4.1} \OperatorTok{*}\StringTok{ }\DecValTok{10}\OperatorTok{^-}\DecValTok{3}
\NormalTok{  speed0 <-}\StringTok{ }\DecValTok{1}
\NormalTok{  tau <-}\StringTok{ }\NormalTok{tau0 }\OperatorTok{*}\StringTok{ }\NormalTok{(mean_speed[segment,]}\OperatorTok{/}\NormalTok{speed0)}\OperatorTok{^}\NormalTok{(}\OperatorTok{-}\FloatTok{0.5}\NormalTok{)}
\NormalTok{  Hf <-}\StringTok{ }\NormalTok{(}\DecValTok{1} \OperatorTok{+}\StringTok{ }\NormalTok{(}\DecValTok{2} \OperatorTok{*}\StringTok{ }\NormalTok{pi }\OperatorTok{*}\StringTok{ }\NormalTok{tau }\OperatorTok{*}\NormalTok{f)}\OperatorTok{^}\DecValTok{2}\NormalTok{)}\OperatorTok{^}\NormalTok{(}\DecValTok{2}\NormalTok{)}
  \CommentTok{# find range where Hf does not exceed 3}
\NormalTok{  ind <-}\StringTok{ }\KeywordTok{length}\NormalTok{(Hf[Hf }\OperatorTok{<=}\StringTok{ }\DecValTok{3}\NormalTok{])}
  \CommentTok{# extract K for range where Hf <= 3}
\NormalTok{  K_max_seg =}\StringTok{ }\NormalTok{waven[ind,segment]}
  
  \CommentTok{# calculate eta (Kolmogorov length scale) from epsilon for check of second criterium for upper k}
\NormalTok{  eta <-}\StringTok{ }\NormalTok{(((}\DecValTok{10}\OperatorTok{^-}\DecValTok{6}\NormalTok{)}\OperatorTok{^}\DecValTok{3}\NormalTok{)}\OperatorTok{/}\KeywordTok{mean}\NormalTok{(eps1[segment],eps2[segment]))}\OperatorTok{^}\FloatTok{0.25} \CommentTok{# Kolmogorov length scale}
  
  \CommentTok{# final upper limit of k}
\NormalTok{  K_max[segment] <-}\StringTok{ }\KeywordTok{min}\NormalTok{(}\FloatTok{0.1}\OperatorTok{/}\NormalTok{eta,K_max_seg)}
\NormalTok{\}}
\end{Highlighting}
\end{Shaded}

\subparagraph{Define x and y for model, only use spectrum for
wavenumbers \textless{}
K\_max}\label{define-x-and-y-for-model-only-use-spectrum-for-wavenumbers-k_max}

\begin{Shaded}
\begin{Highlighting}[]
\CommentTok{# initialise x and y as empty lists}
\NormalTok{x <-}\StringTok{ }\KeywordTok{list}\NormalTok{()}
\NormalTok{y1 <-}\StringTok{ }\KeywordTok{list}\NormalTok{()}
\NormalTok{y2 <-}\StringTok{ }\KeywordTok{list}\NormalTok{()}
\CommentTok{# loop through all spectra for profile 32}
\ControlFlowTok{for}\NormalTok{(segment }\ControlFlowTok{in} \DecValTok{1}\OperatorTok{:}\DecValTok{146}\NormalTok{)\{}
\NormalTok{  K_min <-}\StringTok{ }\DecValTok{0} \CommentTok{# K_min needs to be checked after fitting, default value = 0}
  \CommentTok{# set lower k limit index}
\NormalTok{  min_ind <-}\StringTok{ }\KeywordTok{max}\NormalTok{(}\KeywordTok{length}\NormalTok{(waven[waven[,segment]}\OperatorTok{<=}\NormalTok{K_min, segment]),}\DecValTok{2}\NormalTok{)}
  \CommentTok{# set upper k limit index}
\NormalTok{  max_ind <-}\StringTok{ }\KeywordTok{length}\NormalTok{(waven[waven[,segment]}\OperatorTok{<=}\NormalTok{K_max[segment],segment]) }
  \CommentTok{# define y, y values are corrected wavenumber temp-gradient spectrum --> P_gradT1_c for chosen k range}
\NormalTok{  y1_seg <-}\StringTok{ }\NormalTok{P_gradT1_c[min_ind}\OperatorTok{:}\NormalTok{max_ind,segment]   }
\NormalTok{  y2_seg <-}\StringTok{ }\NormalTok{P_gradT2_c[min_ind}\OperatorTok{:}\NormalTok{max_ind,segment]}
  \CommentTok{# define x, x values are k (wavenumbers) for chosen range}
\NormalTok{  x_seg <-}\StringTok{ }\NormalTok{waven[min_ind}\OperatorTok{:}\NormalTok{max_ind,segment]}
  \CommentTok{# append x and y lists in each step}
\NormalTok{  y1 <-}\StringTok{ }\KeywordTok{c}\NormalTok{(y1,}\KeywordTok{list}\NormalTok{(y1_seg))}
\NormalTok{  y2 <-}\StringTok{ }\KeywordTok{c}\NormalTok{(y2,}\KeywordTok{list}\NormalTok{(y2_seg))}
\NormalTok{  x <-}\StringTok{ }\KeywordTok{c}\NormalTok{(x,}\KeywordTok{list}\NormalTok{(x_seg))}
\NormalTok{\} }\CommentTok{# end for loop}
\CommentTok{# combine y1 and y2 in one list}
\NormalTok{y <-}\StringTok{ }\KeywordTok{list}\NormalTok{(y1, y2)}
\end{Highlighting}
\end{Shaded}

\subparagraph{\texorpdfstring{maximum \(k\) and number of spectral
points included in estimating \(\chi\) for sample
spectra:}{maximum k and number of spectral points included in estimating \textbackslash{}chi for sample spectra:}}\label{maximum-k-and-number-of-spectral-points-included-in-estimating-chi-for-sample-spectra}

\begin{itemize}
\tightlist
\item
  spectrum 16: \(k_{max}\) = 29.5, n = 74
\item
  spectrum 30: \(k_{max}\) = 18.3, n = 44
\item
  spectrum 50: \(k_{max}\) = 15.6, n = 37
\item
  spectrum 70: \(k_{max}\) = 22.1, n = 50
\item
  spectrum 100: \(k_{max}\) = 18.5, n = 39
\end{itemize}

\subparagraph{definition of maximum likelihood
function}\label{definition-of-maximum-likelihood-function}

\begin{Shaded}
\begin{Highlighting}[]
\CommentTok{# define C_T}
\NormalTok{C_T <-}\StringTok{ }\FloatTok{0.4}
\CommentTok{# initialise output fit and chi}
\NormalTok{fit <-}\StringTok{ }\KeywordTok{list}\NormalTok{()}
\NormalTok{chi <-}\StringTok{ }\KeywordTok{c}\NormalTok{()}
\CommentTok{# loop through both temp sensors}
\ControlFlowTok{for}\NormalTok{(sensor }\ControlFlowTok{in} \DecValTok{1}\OperatorTok{:}\DecValTok{2}\NormalTok{)\{}
\NormalTok{  fit_sens <-}\StringTok{ }\KeywordTok{list}\NormalTok{()}
\NormalTok{  chi_sens <-}\StringTok{ }\KeywordTok{c}\NormalTok{()}
  \CommentTok{# loop through each spectrum in profile 32}
  \ControlFlowTok{for}\NormalTok{(segment }\ControlFlowTok{in} \DecValTok{1}\OperatorTok{:}\DecValTok{146}\NormalTok{)\{}
\NormalTok{    xs <-}\StringTok{ }\NormalTok{x[[segment]]}
\NormalTok{    ys <-}\StringTok{ }\NormalTok{y[[sensor]][[segment]]}
    \CommentTok{#likelihood function definition}
\NormalTok{    LL <-}\StringTok{ }\ControlFlowTok{function}\NormalTok{(chi) \{}
      \CommentTok{# Find residuals}
\NormalTok{      R =}\StringTok{ }\NormalTok{ys }\OperatorTok{-}\StringTok{ }\NormalTok{C_T }\OperatorTok{*}\StringTok{ }\NormalTok{chi }\OperatorTok{*}\StringTok{ }\KeywordTok{mean}\NormalTok{(eps1[segment],eps2[segment])}\OperatorTok{^}\NormalTok{(}\OperatorTok{-}\DecValTok{1}\OperatorTok{/}\DecValTok{3}\NormalTok{) }\OperatorTok{*}\StringTok{ }\NormalTok{xs}\OperatorTok{^}\NormalTok{(}\DecValTok{1}\OperatorTok{/}\DecValTok{3}\NormalTok{)}
      \CommentTok{# Calculate the likelihood for the residuals}
\NormalTok{      R =}\StringTok{ }\KeywordTok{suppressWarnings}\NormalTok{(}\KeywordTok{dnorm}\NormalTok{(R, }\DataTypeTok{log =} \OtherTok{TRUE}\NormalTok{))}
      \CommentTok{# Sum the log likelihoods for all of the data points}
      \OperatorTok{-}\KeywordTok{sum}\NormalTok{(R)}
\NormalTok{    \} }\CommentTok{# end of LL function}
  
    \CommentTok{#model fit}
\NormalTok{    fit_seg <-}\StringTok{ }\KeywordTok{suppressWarnings}\NormalTok{(}\KeywordTok{mle}\NormalTok{(LL, }\DataTypeTok{start =} \KeywordTok{list}\NormalTok{( }\DataTypeTok{chi=}\DecValTok{10}\OperatorTok{^}\NormalTok{(}\OperatorTok{-}\DecValTok{7}\NormalTok{))}
\NormalTok{                              ,}\DataTypeTok{nobs =} \KeywordTok{length}\NormalTok{(ys),  }\DataTypeTok{method=}\StringTok{'L-BFGS-B'}\NormalTok{))}
\NormalTok{    fit_sens <-}\StringTok{ }\KeywordTok{c}\NormalTok{(fit_sens, }\KeywordTok{list}\NormalTok{(fit_seg))}
\NormalTok{    chi_sens <-}\StringTok{ }\KeywordTok{c}\NormalTok{(chi_sens, fit_seg}\OperatorTok{@}\NormalTok{coef[[}\DecValTok{1}\NormalTok{]])}
\NormalTok{  \} }\CommentTok{# end for loop through spectra}
\NormalTok{  fit <-}\StringTok{ }\KeywordTok{c}\NormalTok{(fit, }\KeywordTok{list}\NormalTok{(fit_sens))}
\NormalTok{  chi <-}\StringTok{ }\KeywordTok{list}\NormalTok{(chi, chi_sens)}
\NormalTok{\} }\CommentTok{# end for loop through sensors}
\end{Highlighting}
\end{Shaded}

\subparagraph{\texorpdfstring{Estimated \(\chi\) for sample
spectra}{Estimated \textbackslash{}chi for sample spectra}}\label{estimated-chi-for-sample-spectra}

\begin{verbatim}
## [1] "spectrum 16: estimated Chi = 3.25e-07, Eps = 8.92e-09, logLik = -68, AIC = 138"
## [1] "spectrum 30: estimated Chi = 4e-08, Eps = 1.13e-09, logLik = -40.4, AIC = 82.9"
## [1] "spectrum 50: estimated Chi = 3.64e-07, Eps = 5.98e-10, logLik = -34, AIC = 70"
## [1] "spectrum 70: estimated Chi = 6.57e-07, Eps = 2.37e-09, logLik = -45.9, AIC = 93.9"
## [1] "spectrum 100: estimated Chi = 3.07e-07, Eps = 1.18e-09, logLik = -35.8, AIC = 73.7"
## [1] "spectrum 130: estimated Chi = 4.41e-08, Eps = 3.52e-10, logLik = -25.7, AIC = 53.5"
\end{verbatim}

\subparagraph{\texorpdfstring{calculate model with fitted \(\chi\) for
plotting}{calculate model with fitted \textbackslash{}chi for plotting}}\label{calculate-model-with-fitted-chi-for-plotting}

\begin{Shaded}
\begin{Highlighting}[]
\NormalTok{spec_model <-}\StringTok{ }\KeywordTok{list}\NormalTok{()}
\ControlFlowTok{for}\NormalTok{(sensor }\ControlFlowTok{in} \DecValTok{1}\OperatorTok{:}\DecValTok{2}\NormalTok{)\{}
\NormalTok{  spec_model_sens <-}\StringTok{ }\KeywordTok{list}\NormalTok{()}
  \ControlFlowTok{for}\NormalTok{(segment }\ControlFlowTok{in} \DecValTok{1}\OperatorTok{:}\DecValTok{146}\NormalTok{)\{}
\NormalTok{    spec_model_seg <-}\StringTok{  }\NormalTok{C_T }\OperatorTok{*}\StringTok{ }\NormalTok{fit[[sensor]][[segment]]}\OperatorTok{@}\NormalTok{coef[}\DecValTok{1}\NormalTok{] }\OperatorTok{*}\StringTok{ }\KeywordTok{mean}\NormalTok{(eps1[segment],eps2[segment])}\OperatorTok{^}\NormalTok{(}\OperatorTok{-}\DecValTok{1}\OperatorTok{/}\DecValTok{3}\NormalTok{) }\OperatorTok{*}\StringTok{ }\NormalTok{x[[segment]]}\OperatorTok{^}\NormalTok{(}\DecValTok{1}\OperatorTok{/}\DecValTok{3}\NormalTok{)}
\NormalTok{    spec_model_sens <-}\StringTok{ }\KeywordTok{c}\NormalTok{(spec_model_sens, }\KeywordTok{list}\NormalTok{(spec_model_seg))}
\NormalTok{  \}}\CommentTok{#end for loop through spectra}
\NormalTok{  spec_model <-}\StringTok{ }\KeywordTok{c}\NormalTok{(spec_model, }\KeywordTok{list}\NormalTok{(spec_model_sens))}
\NormalTok{\}}\CommentTok{#end for loop through sensors}
\end{Highlighting}
\end{Shaded}

\subparagraph{plot data and model for sample
spectra}\label{plot-data-and-model-for-sample-spectra}

\begin{Shaded}
\begin{Highlighting}[]
\NormalTok{pl_model <-}\StringTok{ }\KeywordTok{data.frame}\NormalTok{()}
\ControlFlowTok{for}\NormalTok{(segment }\ControlFlowTok{in}\NormalTok{ specnr_sub)\{}
\NormalTok{  k_model_s <-}\StringTok{ }\NormalTok{x[[segment]]}
\NormalTok{  spec_model1_s <-}\StringTok{ }\NormalTok{spec_model[[}\DecValTok{1}\NormalTok{]][[segment]]}
\NormalTok{  spec_model2_s <-}\StringTok{ }\NormalTok{spec_model[[}\DecValTok{2}\NormalTok{]][[segment]]}
\NormalTok{  pl_model <-}\StringTok{ }\NormalTok{pl_model }\OperatorTok
\StringTok{      }\KeywordTok{rbind}\NormalTok{(}\KeywordTok{data.frame}\NormalTok{(}\DataTypeTok{k_model =}\NormalTok{ k_model_s, }\DataTypeTok{T1 =}\NormalTok{ spec_model1_s, }\DataTypeTok{T2 =}\NormalTok{ spec_model2_s, }\DataTypeTok{specnr =} \KeywordTok{as.factor}\NormalTok{(segment)) }\OperatorTok
\StringTok{    }\KeywordTok{gather}\NormalTok{(}\DataTypeTok{key =}\NormalTok{ sensor, }\OperatorTok{-}\NormalTok{k_model, }\OperatorTok{-}\NormalTok{specnr, }\DataTypeTok{value =}\NormalTok{ spec_model))}
\NormalTok{\} }\CommentTok{#end for loop through spectra}
\NormalTok{pl_data <-}\StringTok{ }\NormalTok{gradT_df }\OperatorTok
\StringTok{  }\KeywordTok{filter}\NormalTok{(specnr }\OperatorTok\StringTok{ }\NormalTok{specnr_sub, freq }\OperatorTok{!=}\StringTok{ }\DecValTok{0}\NormalTok{) }\OperatorTok
\StringTok{  }\KeywordTok{mutate}\NormalTok{(}\DataTypeTok{specnr =} \KeywordTok{as.factor}\NormalTok{(specnr))}
\KeywordTok{ggplot}\NormalTok{() }\OperatorTok{+}\StringTok{ }
\StringTok{  }\KeywordTok{geom_point}\NormalTok{(}\DataTypeTok{data=}\NormalTok{pl_data, }\KeywordTok{aes}\NormalTok{(}\DataTypeTok{x=}\NormalTok{freq, }\DataTypeTok{y =}\NormalTok{ gradT_c, }\DataTypeTok{col =}\NormalTok{ sensor), }\DataTypeTok{alpha =} \FloatTok{0.3}\NormalTok{) }\OperatorTok{+}\StringTok{ }
\StringTok{  }\KeywordTok{geom_line}\NormalTok{(}\DataTypeTok{data=}\NormalTok{pl_model, }\KeywordTok{aes}\NormalTok{(}\DataTypeTok{x=}\NormalTok{k_model, }\DataTypeTok{y=}\NormalTok{spec_model, }\DataTypeTok{col =}\NormalTok{ sensor)) }\OperatorTok{+}
\StringTok{  }\KeywordTok{scale_y_continuous}\NormalTok{(}\DataTypeTok{trans=}\StringTok{'log10'}\NormalTok{, }\DataTypeTok{limits=}\KeywordTok{c}\NormalTok{(}\DecValTok{10}\OperatorTok{^-}\DecValTok{9}\NormalTok{,}\DecValTok{10}\OperatorTok{^-}\DecValTok{2}\NormalTok{)) }\OperatorTok{+}
\StringTok{  }\KeywordTok{scale_x_continuous}\NormalTok{(}\DataTypeTok{trans=}\StringTok{'log10'}\NormalTok{)  }\OperatorTok{+}\StringTok{ }
\StringTok{  }\KeywordTok{labs}\NormalTok{(}\DataTypeTok{x =} \StringTok{'freq'}\NormalTok{, }\DataTypeTok{y =} \StringTok{'spectra energy'}\NormalTok{) }\OperatorTok{+}
\StringTok{  }\KeywordTok{facet_wrap}\NormalTok{(}\OperatorTok{~}\NormalTok{specnr) }\OperatorTok{+}
\StringTok{  }\KeywordTok{theme_bw}\NormalTok{()}
\end{Highlighting}
\end{Shaded}

\includegraphics{temp_spectra_examples_files/figure-latex/unnamed-chunk-11-1.pdf}

\subparagraph{\texorpdfstring{check for lower \(k\) condition using MAD
(mean absolute deviation) between observed and modelled
spectra}{check for lower k condition using MAD (mean absolute deviation) between observed and modelled spectra}}\label{check-for-lower-k-condition-using-mad-mean-absolute-deviation-between-observed-and-modelled-spectra}

\begin{Shaded}
\begin{Highlighting}[]
\CommentTok{# only checking T1 for now}
\NormalTok{MAD <-}\StringTok{ }\KeywordTok{list}\NormalTok{()}
\NormalTok{crit <-}\StringTok{ }\KeywordTok{list}\NormalTok{()}
\NormalTok{spec_good <-}\StringTok{ }\KeywordTok{c}\NormalTok{()}
  \ControlFlowTok{for}\NormalTok{(segment }\ControlFlowTok{in} \DecValTok{1}\OperatorTok{:}\DecValTok{146}\NormalTok{)\{}
\NormalTok{    phi <-}\StringTok{ }\NormalTok{y[[}\DecValTok{1}\NormalTok{]][[segment]]   }\CommentTok{# corrected wavenumber temp-gradient = measured spectrum}
\NormalTok{    psi <-}\StringTok{ }\NormalTok{spec_model[[}\DecValTok{1}\NormalTok{]][[segment]]   }\CommentTok{# modelled spectrum}
\NormalTok{    steps <-}\StringTok{ }\DecValTok{6}  \CommentTok{# 0.5-0.7 of a decade long?}
\NormalTok{    nr_steps <-}\StringTok{ }\KeywordTok{floor}\NormalTok{(}\KeywordTok{length}\NormalTok{(phi)}\OperatorTok{/}\NormalTok{steps)}
\NormalTok{    modulo <-}\StringTok{ }\KeywordTok{length}\NormalTok{(phi) }\OperatorTok\StringTok{ }\NormalTok{steps}
    \ControlFlowTok{if}\NormalTok{(modulo }\OperatorTok{>=}\StringTok{ }\DecValTok{4}\NormalTok{)\{}
\NormalTok{      nr_steps <-}\StringTok{ }\NormalTok{nr_steps }\OperatorTok{+}\StringTok{ }\DecValTok{1} 
\NormalTok{    \}}

    \CommentTok{# calculate MAD for steps of 6 values for the spectrum}
\NormalTok{    MAD_seg <-}\StringTok{ }\KeywordTok{c}\NormalTok{()}
\NormalTok{    crit_seg <-}\StringTok{ }\KeywordTok{c}\NormalTok{()}
    \ControlFlowTok{for}\NormalTok{(i }\ControlFlowTok{in} \DecValTok{1}\OperatorTok{:}\NormalTok{nr_steps)\{}
\NormalTok{      n_st <-}\StringTok{ }\NormalTok{(i}\OperatorTok{-}\DecValTok{1}\NormalTok{) }\OperatorTok{*}\StringTok{ }\NormalTok{steps }\OperatorTok{+}\StringTok{ }\DecValTok{1}
      \ControlFlowTok{if}\NormalTok{(i }\OperatorTok{==}\StringTok{ }\NormalTok{nr_steps)\{}
        \ControlFlowTok{if}\NormalTok{(modulo }\OperatorTok{>=}\StringTok{ }\DecValTok{4}\NormalTok{)\{}
\NormalTok{          n_en <-}\StringTok{ }\NormalTok{n_st }\OperatorTok{+}\StringTok{ }\NormalTok{modulo }\OperatorTok{-}\StringTok{ }\DecValTok{1}
\NormalTok{        \}}\ControlFlowTok{else}\NormalTok{\{}
\NormalTok{          n_en <-}\StringTok{ }\NormalTok{n_st }\OperatorTok{+}\StringTok{ }\NormalTok{steps }\OperatorTok{-}\StringTok{ }\DecValTok{1} \OperatorTok{+}\StringTok{ }\NormalTok{modulo}
\NormalTok{        \}}
\NormalTok{      \}}\ControlFlowTok{else}\NormalTok{\{}
\NormalTok{        n_en <-}\StringTok{ }\NormalTok{n_st }\OperatorTok{+}\StringTok{ }\NormalTok{steps }\OperatorTok{-}\StringTok{ }\DecValTok{1}
\NormalTok{      \}}
\NormalTok{      phi_n =}\StringTok{ }\NormalTok{phi[n_st}\OperatorTok{:}\NormalTok{n_en]}
\NormalTok{      psi_n =}\StringTok{ }\NormalTok{psi[n_st}\OperatorTok{:}\NormalTok{n_en]}
\NormalTok{      MAD_seg[i] <-}\StringTok{  }\DecValTok{1}\OperatorTok{/}\NormalTok{(}\KeywordTok{length}\NormalTok{(phi_n)) }\OperatorTok{*}\StringTok{ }\KeywordTok{sum}\NormalTok{(}\KeywordTok{abs}\NormalTok{(phi_n}\OperatorTok{/}\NormalTok{psi_n }\OperatorTok{-}\StringTok{ }\KeywordTok{mean}\NormalTok{(phi_n}\OperatorTok{/}\NormalTok{psi_n)))}
    
      \CommentTok{# check whether MAD fits criteria, crit = 0 (bad), 1 (good)}
\NormalTok{      dof <-}\StringTok{ }\KeywordTok{length}\NormalTok{(phi_n) }\OperatorTok{-}\StringTok{ }\DecValTok{1}   \CommentTok{# degrees of freedom}
\NormalTok{      crit_seg[i] <-}\StringTok{ }\KeywordTok{ifelse}\NormalTok{(MAD_seg[i] }\OperatorTok{<}\StringTok{ }\DecValTok{2} \OperatorTok{*}\StringTok{ }\NormalTok{(}\DecValTok{2}\OperatorTok{/}\NormalTok{dof)}\OperatorTok{^}\FloatTok{0.5}\NormalTok{, }\DecValTok{1}\NormalTok{, }\DecValTok{0}\NormalTok{)}
    
\NormalTok{    \} }\CommentTok{# end for loop}
\NormalTok{    MAD <-}\StringTok{ }\KeywordTok{c}\NormalTok{(MAD, }\KeywordTok{list}\NormalTok{(MAD_seg))}
\NormalTok{    crit <-}\StringTok{ }\KeywordTok{c}\NormalTok{(crit, }\KeywordTok{list}\NormalTok{(crit_seg))}
\NormalTok{    spec_good <-}\StringTok{ }\KeywordTok{c}\NormalTok{(spec_good, }\KeywordTok{ifelse}\NormalTok{(}\KeywordTok{sum}\NormalTok{(crit_seg)}\OperatorTok{==}\KeywordTok{length}\NormalTok{(crit_seg),}\DecValTok{1}\NormalTok{,}\DecValTok{0}\NormalTok{))}
\NormalTok{  \}}
  \CommentTok{#If all subsets of the spectrum yielded a MAD>2(2/d)^1/2 (d = degrees of freedom   --> n-1 ?) (Ruddick et al. 2000), then the spectrum is completely discarded. Otherwise, the final wavenumber range used to obtain xF starts at the lowest wavenumber at which MAD<2(2/d)^1/2}
\end{Highlighting}
\end{Shaded}

There are 70 good spectra out of 146 spectra in the profile.

\subparagraph{For our sample spectra:}\label{for-our-sample-spectra}

\begin{verbatim}
## [1] "spectrum 16: all MADs fulfill criteria, good spectrum"
## [1] "spectrum 30: at least one MAD value outside allowed range, change minimum k"
## [1] "spectrum 50: at least one MAD value outside allowed range, change minimum k"
## [1] "spectrum 70: all MADs fulfill criteria, good spectrum"
## [1] "spectrum 100: all MADs fulfill criteria, good spectrum"
## [1] "spectrum 130: all MADs fulfill criteria, good spectrum"
\end{verbatim}


\end{document}
